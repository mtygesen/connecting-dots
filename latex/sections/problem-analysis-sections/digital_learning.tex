\section{Digital Learning}\label{ch:digital_learning_ch}
\\Digital learning - Introduktion til kapitlet, hvorfor skal vi vide det her i forhold til problemet?
\subsection{What is digital learning?}\label{ch:what_is_digital_learning}
\subsection{Visual learning and its properties}\label{ch:visual_learning_and_its_properties}
\subsection{Spatial intelligence and awareness}\label{ch:spatial_intelligence_and_awareness}
The capacity to imagine and visualize different objects and patterns is one of the 9 intelligences, some people prefer a hands on approach, need to ask questions in order to learn, others need to write things down and most of us can gain a quicker understanding through visual learning.\cite{pracpsych2022}.

(Lohman 1996) states that spatial intelligence, or visuo-spatial ability, has been defined as "the ability to generate, retain, retrieve, and transform well-structured visual images”\cite[p97]{tapsfield1996}, with that definition in mind this section will describe the importance of spatial intelligence in regards to visual learning and spatial reasoning.

\\læringsstile - kort intro
\\fotografisk hukommelse
\\Spatial visualizeres & object visualizers

\subsection{The ability to visualize}\label{ch:the_ability_to_visualize}
We probably all have heard the saying "a picture is worth a thousand words", this saying was originally invented by and advertising executive (Fred R. Barnard)\cite{phrases2022}, like in advertising the power of the retainability of the message is in focus. 

\\Need to see it to believe it (doctor - patient relation)
\\farveblinde, ordblinde

\\Menneskets evne og/ellers computers evne til at formidle visuelt 
\\A geoscientist mentally manipulates the movement of tectonic plates to see the process of earth formation.
\\A neurosurgeon visualizes different brain areas to predict the outcome of a surgery.
\\A civil engineer imagines how various forces may affect the design of a system.
\\Architects and engineers use material of various shapes and sizes to create stable structures.
\\A designer uses visual spatial reasoning concept to enhance the user experience of his product.
\\An artist creates stunning visual arts.
\\A gymnast uses spatial awareness to perform a sequence of movements with the human body.

There are many benefits to visual representation, some of these benefits will be reviewed in the next section.
\subsection{Visual and text representation}\label{ch:visual_and_text_representation}
\\Nemmere ved at repræsentere forskelle, medianer, yder punkter, statistikker, grafer osv. osv.
\\Matematik -> geometri, patterns, arithmetic, number sense osv.


\\OUTRO
\\Hurtig konklusion og oplæg til næste kapitel
\\Kun nogle af os kan have nytte af dette
\\Sætter nogle krav til visualiseringen -> oplæg til machine learning?
\\EARLY learning -> evt. bruges til projektforslag (ellers evt. skip)

\section{Opbygning - slet senere}\label{ch:opbygning}

What is digital learning? - Hvad består begrebet digital learning af og hvad er de primære elementer i det? 
    Hvem bruger det og hvem gavner det? 
    Hvilke værktøjer benytter man til den digitale proces?

  Visual learning and its properties - Den digitale proces er visuelt repræsenteret
    kræver mindre interaktion, der står en del om forskellen på e-learning og digital learning somewhere
    fordele/ulemper ved visuel læring generelt (masser af keywords nedenfor), husk at skelne generelt og mennesket

Keywords/sentences regarding digital learning: 
  Pros -> Learn from anywhere, anytime, learn in your own pace, revisit materials and redo lectures, track progress, gamifiction, personalized learning, supports blended learning, collaborative learning.

  Cons -> Limited teacher & student interaction, risk of social isolation, cheating is more brutal to monitor, digitally challenged people has trouble, issues with quality, requires self-motivation and time management, practical component of learning suffers, limited to specific disciplines, lack of development of communication skills, prolonged screen exposure.